% Defaults to one column, but has a twocolumn option
%\documentclass[twocolumn]{cicirello}
\documentclass{cicirello}
% Keep in first 5 lines of tex file in support of submitting to arXiv 
\pdfoutput=1

\usepackage{lipsum}

\title{Example of How To Use the Class}
\author{Firstname M. Lastname}
\orcid{0000-0000-0000-0000}
\reportnum{VAC-TR-00-0000}
\reporturl{https://reports.cicirello.org/}
\copyrightyear{2024}
\date{January 2024}

% Use this if the report has been published somewhere such as in a journal or conference.
\citeas{Firstname M. Lastname. 2024. Example of How To Use the Class. \textit{Title of the Journal}, 42(42): 100--142, January 2024. doi:10.5555/555.555555.55555}
% Use this to specify type of citation. it defaults to: Journal ref
%\citetype{}

\address{Computer Science\\
Stockton University\\
Galloway, NJ 08205 USA\\
\url{https://www.cicirello.org/}
}

%% Set these here before \begin{document}
%% All of these are optional.
\keywords{hello, world, goodbye, sky}
\ACM{I.2.8; F.2.2; G.2.1}
\MSC{68W50}

%% Use this to set the Subject property within the
%% pdf metadata. This is optional. This does not 
%% appear in the pdf content itself. For search
%% engine's use should probably be at most a couple
%% hundred characters in length. Think of it as a very
%% short version of abstract.
\subject{Here is an abridged version of my abstract.}

%% My Tech Report class redefines abstract as a command
%% so put here (abstract no longer an environment).
\abstract{\lipsum[1]}


\begin{document}

\maketitle

\section{Hello}

\lipsum[2-6]

\section{World}

Here is another section.

\end{document}